\section{Top quark and W boson}

\begin{enumerate}
  \item
	The CKM-matrix is a unitary matrix where each element holds information about the
      	strength of the flavour changing weak deacys which happens between quarks.
      	These changes are mediated with the $W^{\pm}$ boson.
      	When four quarks were discovered it was created two sets of equation describing
      	the decay from down and strange into top and charm.
      	Seeing that with CP-violation could not be explained with these four quarks,
      	they added another generation to create the CKM-matrix:
      	\begin{equation}
      	\begin{bmatrix}
      	  d'\\
      	  s'\\
      	  b'
      	\end{bmatrix}
      	=
      	\begin{bmatrix}
      	  V_{ud} & V_{us} & V_{ub} \\
      	  V_{cd} & V_{cs} & V_{cb} \\
      	  V_{td} & V_{ts} & V_{tb} 
      	\end{bmatrix}
      	\begin{bmatrix}
      	  d\\
      	  s\\
      	  b
      	\end{bmatrix}
      	\end{equation}
      	The $W$ boson is a mediater of the weak force, it has either $+1$ or $-1$ charge so it
	can react with charged particles. In regards to the CKM-matrix it is the mediator for
	decaying quarks between up and down types as well as changing flavours.
    \begin{enumerate}
      \item
	$V_{ud}$ can be experimentally shown from the ration between netron decay and
	$\mu$ decay.\\
	$V_{us}$ is shown in the $K^+ \rightarrow \pi^0 e^+ \nu_e$ decay proceAss.\\
	$V_{cs}$ is experimentally shown in hadronic decays of $W^{\pm}$ and 
	$D\rightarrow \bar{K}e^+\nu_e$ process.
    \end{enumerate}
  \item
    \begin{enumerate}
      \item
	For the electron-positron annihilation, the $e^-e^+ \rightarrow W^- t \bar{b}$ 
	process requires high energy input because of the large difference in masses
	from starting to resulting particles. Which is more difficult for linear colliders
	to achieve.\\
	For the proton-proton collision these energies are easier to reach because of the circular
	colliders.
	The proton-antiproton collisions provide even more energies due to the annihilation of
	particles, but experimentally harder to accelerate to high energies.
      \item
    \end{enumerate}
  \item
    \begin{enumerate}
      \item
      \item
	\begin{enumerate}
	  \item
	  \item
	\end{enumerate}
      \item
	\begin{enumerate}
	  \item
	\end{enumerate}
    \end{enumerate}
  \item
\end{enumerate}
