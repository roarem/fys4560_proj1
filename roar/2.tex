\section{Top quark and W boson}
  \subsection{CKM and W-boson}
	The CKM-matrix is a unitary matrix where each element holds information about the
      	strength of the flavour changing weak deacys which happens between quarks.
      	These changes are mediated with the $W^{\pm}$ boson.
      	When four quarks were discovered it was created two sets of equation describing
      	the decay from down and strange into top and charm.
      	Seeing that with CP-violation could not be explained with these four quarks,
      	they added another generation to create the CKM-matrix:
      	\begin{equation}
      	\begin{bmatrix}
      	  d'\\
      	  s'\\
      	  b'
      	\end{bmatrix}
      	=
      	\begin{bmatrix}
      	  V_{ud} & V_{us} & V_{ub} \\
      	  V_{cd} & V_{cs} & V_{cb} \\
      	  V_{td} & V_{ts} & V_{tb} 
      	\end{bmatrix}
      	\begin{bmatrix}
      	  d\\
      	  s\\
      	  b
      	\end{bmatrix}
      	\end{equation}
      	The $W$ boson is a mediater of the weak force, it has either $+1$ or $-1$ charge so it
	can react with charged particles. In regards to the CKM-matrix it is the mediator for
	decaying quarks between up and down types as well as changing flavours.
	$V_{ud}$ can be experimentally shown from the ratio between netron decay and
	$\mu$ decay.\\
	$V_{us}$ is shown in the $K^+ \rightarrow \pi^0 e^+ \nu_e$ decay process.\\
	$V_{cs}$ is experimentally shown in hadronic decays of $W^{\pm}$ and 
	$D\rightarrow \bar{K}e^+\nu_e$ process.

  \subsection{Top quark production}
    \begin{itemize}
      \item
	In an $e^-e_+$ anihilation there will be produced a $\gamma$ or $Z$
	which in turn can decay into a $q\bar{q}$. The quark pair could be $t\bar{t}$,
	but because of its high mass and the relativistic speed the electrons must reach,
	this process will be supressed
      \item
	As for the $pp$-collison there are alot that can happen. Gluons, $g$, can be produced which 
	in turn creates a $t\bar{t}$-pair $g \rightarrow t\bar{t}$, and $g$ going into $b\bar{b}$ in 
	which one of the $b$s interacts with a $W$ to become a singel $t$, $g \rightarrow b\bar{b} %
	\rightarrow Wb t$.
      \item
	When a $p\bar{p}$ annihilation occurs there will be produced high energy gluons
	that create $t\bar{t}$, and $b\bar{b}$ to $W$ and $bt$ for a single top. Same as 
	a $pp$ collison but with more energy leftover to produce the top pairs.
    \end{itemize}

    \begin{figure}[ht]
      \centering
      \begin{subfigure}[b]{0.5\textwidth}
        \includegraphics[width=\textwidth]{singleTop.pdf}
	\caption{$pp \rightarrow W^+\bar{t}b$}
        \label{top:1}
      \end{subfigure}%
      ~
      \begin{subfigure}[b]{0.5\textwidth}
        \includegraphics[width=\textwidth]{topAntiTop.pdf}
	\caption{$pp \rightarrow \bar{t}t$}
        \label{top:2}
      \end{subfigure}
    \end{figure}

  \subsection{Top quark decay}

    The top quark is very heavy compared to the other elementary particles in the standard model.
    Due to this mass it only as a lifetime of $5\times 10^{-25}\,s$, and therefore do not have time
    to create hadrons, and is the only quark we can observe alone in some sense. 
    Because of that it deacys into a $W$ and a bottom, strange or down quark, which is also the only 
    observed decay mode of the $t$ quark. The branching ratio of it decaying into a $b$ compared
    to the other quarks are about $99\%$.

    The $t$ decays therefore into a $W$ boson which means it can have a leptonic or hadronic final
    state.
    \newpage
    \begin{figure}[ht]
      \centering
      \begin{subfigure}[b]{0.5\textwidth}
        \includegraphics[width=\textwidth]{topDecayLep.pdf}
	\caption{$t \rightarrow bl^+\nu_l$}
        \label{top:3}
      \end{subfigure}%
      ~
      \begin{subfigure}[b]{0.5\textwidth}
        \includegraphics[width=\textwidth]{topDecayHad.pdf}
	\caption{$t \rightarrow b\bar{q}q$}
        \label{top:4}
      \end{subfigure}
    \end{figure}
    \begin{flalign} 
      &t \rightarrow W^+b \rightarrow q\bar{q} b\\
      &\bar{t} \rightarrow W^-\bar{b} \rightarrow l^-\bar{\nu} \bar{b}
    \end{flalign}
  
    \begin{figure}[ht]
      \centering
      \includegraphics[scale=0.09]{singleTop.png}
      \caption{An event candidate for single top quark. Top decaying to electron (red line),%
      two jets (blue and yellow), and several particles that can not be detected where the missing%
      energy is clumped together (pink line). Picture gotten from %
      \href{http://www.atlas.ch/news/2011/eps-summary.html}{Atlas link}%
    }
    \end{figure}

  \subsection{Branching ratios of top decays}
    \[ W_i = \frac{\Gamma_i/\Gamma}{\tau} \approx G_F (\Delta m)^5\]
    $G_F \approx 1.17 \times 10^{-5}\,GeV^{-2}$\\
    The branching ratio will be about the size of the mass difference in the process.
    \[m_t \approx 173\,GeV,\,m_b \approx 4\,GeV,\, m_c \approx 1\,GeV,\, m_s \approx %
      95\,MeV,\, m_{\tau} \approx 1776\,MeV \approx 2\,GeV\]
    $t\rightarrow b+c\bar{s}$
    \[ \Delta m = m_t - m_b - m_c - m_s  \approx  168\,GeV\]
    $t\rightarrow b + \tau^+\nu_{\tau}$ aproximate the neutrino to be massless.
    \[ \Delta m = m_t - m_b - m_{\tau^+}  \approx  167\,GeV\]
   \newpage 
